\documentclass[conference]{IEEEtran}

\usepackage{cite}
\usepackage{amsmath,amssymb,amsfonts}
\usepackage{algorithmic}
\usepackage{graphicx}
\usepackage{textcomp}
\usepackage{xcolor}
\usepackage{url}
\usepackage{booktabs}
\usepackage{multirow}
\usepackage{placeins}
\usepackage{caption}


\begin{document}

\title{5G NR Multi-UE Paging System with Real-Time Monitoring and ASN.1 Protocol Validation}

\author{\IEEEauthorblockN{Sritama Biswas}
\IEEEauthorblockA{\textit{FWC COMET Foundation} \\
\textit{IIIT Bangalore}\\
Bangalore, India \\
sritamabiswas.fwc1@iiitb.ac.in}}

\maketitle

\begin{abstract}
Paging is a fundamental control-plane procedure in 5G New Radio (NR) networks that enables efficient device reachability while minimizing power consumption. This paper presents the design, implementation, and analysis of a multi-UE paging system in a 5G NR Standalone (SA) network, realized entirely in software without radio frequency hardware. The system integrates Open5GS core network, srsRAN Project software-defined gNB, and three independent UE instances interconnected through ZeroMQ. A custom ASN.1 Packed Encoding Rules (PER) decoder validates protocol correctness at the bit level. Experimental results demonstrate 100\% paging selectivity across multiple UEs, accurate RRC state transitions, and paging latency of approximately 8.3ms, comparable to commercial deployments. The platform provides a reproducible, research-grade testbed for studying 5G control-plane behavior and serves as an accessible educational tool for protocol analysis.
\end{abstract}

\begin{IEEEkeywords}
5G NR, Paging, Multi-UE, ASN.1, NGAP, RRC, Open5GS, srsRAN, Software-Defined Radio, Protocol Validation
\end{IEEEkeywords}

\section{Introduction}

The Fifth Generation (5G) cellular system represents a paradigm shift in mobile network architecture, emphasizing not only enhanced throughput but also architectural flexibility, energy efficiency, and service differentiation through a modularized Service-Based Architecture (SBA) \cite{3gpp_arch}. Within this evolved ecosystem, paging emerges as a critical control-plane procedure that enables User Equipments (UEs) to remain reachable while in low-power idle states.

In 5G NR, paging complexity increases significantly compared to LTE due to flexible numerology, beam-based transmission in massive MIMO systems, enhanced identity management structures (5G-S-TMSI), and tighter coupling between Access and Mobility Management Function (AMF), NG Application Protocol (NGAP) signaling, Radio Resource Control (RRC) state management, and tracking area configuration \cite{3gpp_procedures}. Any implementation flaw in this multi-layer paging chain can severely impact network scalability, reliability, and user experience.

\subsection{Research Challenges}

Commercial 5G networks present significant barriers to detailed paging analysis: (1) proprietary implementations restrict visibility into internal signaling mechanisms, (2) traditional experimentation requires expensive Software-Defined Radio (SDR) equipment and licensed spectrum, (3) most academic demonstrations remain limited to single-UE scenarios with minimal protocol observability, and (4) absence of standardized tools for ASN.1-level protocol verification in research environments.

\subsection{Contributions}

This work addresses these challenges through the following key contributions:

\begin{itemize}
\item A fully virtualized multi-UE 5G NR paging environment that preserves authentic protocol behavior while eliminating RF hardware dependencies
\item Comprehensive monitoring mechanisms spanning multiple protocol layers (NAS, NGAP, RRC, MAC, PHY) with real-time event correlation
\item A custom ASN.1 PER encoder/decoder framework based on official 3GPP specifications for rigorous protocol-level verification
\item Demonstration and validation of selective paging behavior across three independent UE instances with strict isolation
\item Performance characterization showing paging latency comparable to commercial deployments
\end{itemize}

The platform extends beyond functional demonstration to provide the deep observability and standards compliance necessary for serious research and development in 5G control-plane mechanisms.

\section{5G NR Paging Fundamentals}

\subsection{RRC State Management}

In 5G NR, UE connectivity is governed by the RRC protocol, which defines three primary states: RRC\_IDLE (no RRC connection, performs cell selection and monitors paging), RRC\_INACTIVE (maintains RAN context in low-power state), and RRC\_CONNECTED (active connection with dedicated resources) \cite{3gpp_rrc}. This work focuses on RRC\_IDLE to RRC\_CONNECTED transitions triggered by paging, representing the most common scenario in practical deployments.

\subsection{End-to-End Paging Flow}

The complete paging procedure involves coordination across multiple entities: (1) AMF maintains UE registration state and initiates paging when downlink data notification arrives from Session Management Function (SMF), (2) NGAP carries paging requests from AMF to gNB over SCTP transport with ASN.1 PER encoding, (3) gNB translates NGAP paging into RRC messages and schedules them on the Paging Control Channel (PCCH), and (4) UE monitors paging occasions, checks identity match, and initiates Random Access if paged \cite{3gpp_ngap}.

\subsection{Paging Message Structure}

The NGAP Paging message contains critical fields as defined in 3GPP TS 38.413: UE Paging Identity (5G-S-TMSI consisting of AMF Set ID, AMF Pointer, and 5G-TMSI), TAI List for Paging, optional Paging DRX, Paging Priority, and Assistance Data. The 5G-S-TMSI structure enables efficient paging selectivity where only the intended UE responds to the broadcast message.

\subsection{Paging Occasion Calculation}

The UE monitors specific subframes called Paging Occasions (PO) determined by:
\begin{equation}
PF = (T/N) \times (UE\_ID \mod N)
\end{equation}
where $T$ is the DRX cycle (32, 64, 128, 256 frames), $N$ is the number of paging frames per DRX cycle, and $UE\_ID$ is derived from 5G-S-TMSI. This deterministic calculation ensures distributed paging load while minimizing UE monitoring overhead.

\section{System Architecture}

\subsection{Overall Design}

The complete system is deployed on a single Ubuntu 22.04 LTS workstation, leveraging virtualization and software-defined radio concepts. Fig. \ref{fig:arch} illustrates the architecture comprising Open5GS 5G Core, srsRAN Project gNB, and three independent UE instances interconnected through ZeroMQ.

\begin{figure}[t]
    \centering
    \includegraphics[width=\columnwidth]{architecture.png}
    
    \caption{Architecture of paging flow}
    \label{fig:arch_paging}
\end{figure}


\subsection{5G Core Network}

Open5GS (version 2.7.0+) provides complete 5G Core Network Functions including AMF, SMF, User Plane Function (UPF), and Network Repository Function (NRF). The AMF manages UE registration and implements paging logic with PLMN 999/70, TAC 1, and NGAP endpoint at 127.0.0.5:38412. Three UE subscriptions are provisioned with unique IMSIs (999700000000001-003) and allocated IP addresses from pool 10.45.0.0/16.

\subsection{gNB Implementation}

The gNB uses srsRAN Project (version 24.04+) operating in ZeroMQ mode, which replaces physical RF interfaces with software message queues while preserving complete PHY/MAC/RLC/RRC protocol stack behavior. Key parameters include PCI 1, TAC 1, band n78 (3.5 GHz), 20 MHz bandwidth, and ZeroMQ transport at tcp://127.0.0.1:2000-2001. This configuration eliminates RF hardware dependencies while maintaining full 3GPP protocol compliance.

\subsection{Multi-UE Framework}

To achieve true multi-UE behavior with strict isolation, each UE instance operates within a dedicated Linux network namespace. Each srsUE process is assigned unique IMSI, dedicated ZeroMQ ports, and separate TUN interfaces (tun\_srsue1-3). This design ensures complete network stack isolation, independent routing tables, prevention of cross-UE traffic leakage, and realistic multi-user scenario emulation.

\subsection{Data Plane Configuration}

The complete data plane path involves UE traffic on TUN interface, tunneling through PDCP/RLC/MAC to gNB, forwarding via N3 interface (GTP-U) to UPF, and routing to destination via ogstun interface. When a UE enters RRC\_IDLE and downlink traffic arrives, SMF triggers paging through AMF, initiating the paging procedure under study.

\section{ASN.1 Protocol Validation Framework}

\subsection{Motivation}

While log files and packet captures provide valuable insights, they do not guarantee bit-level protocol correctness. Direct ASN.1 decoding enables verification of correct PER encoding, validation of field values against specifications, detection of subtle encoding errors, and cross-verification with Wireshark dissections \cite{asn1_per}.

\subsection{Implementation Approach}

Official 3GPP ASN.1 modules from TS 38.413 (NGAP) and TS 38.331 (NR-RRC) are compiled using asn1c compiler with PER encoding support \cite{asn1c}. The generated C code includes structure definitions, PER encoding/decoding functions, and constraint checking. The decoder extracts paging messages from PCAP files, decodes ASN.1 structures, validates 5G-S-TMSI components (AMF Set ID, AMF Pointer, 5G-TMSI), and verifies TAI list encoding.

\subsection{Validation Workflow}

The complete validation consists of: (1) capturing NGAP traffic during paging tests using tcpdump, (2) parsing PCAP to isolate paging messages, (3) applying ASN.1 decoder to extract structured data, (4) comparing decoded values with expected UE identities, and (5) cross-checking against Wireshark dissections and gNB logs. This multi-source validation provides stronger experimental rigor than single-source methods.

\vspace{0.5em}
\begin{center}
    \includegraphics[width=\columnwidth]{core.jpg}
    \captionof{figure}{srsRAN core running terminal}
    \label{fig:asn1_validation}
\end{center}

\vspace{0.5em}
\begin{center}
    \includegraphics[width=\columnwidth]{gnb.jpg}
    \captionof{figure}{gNB terminal}
    \label{fig:asn1_validation}
\end{center}

\begin{center}
    \includegraphics[width=\columnwidth]{ue1.jpg}
    \captionof{figure}{UE1 terminal}
    \label{fig:asn1_validation}
\end{center}

\begin{center}
    \includegraphics[width=\columnwidth]{ue2.jpg}
    \captionof{figure}{UE2 terminal}
    \label{fig:asn1_validation}
\end{center}

\begin{center}
    \includegraphics[width=\columnwidth]{ue3.jpg}
    \captionof{figure}{UE3 terminal}
    \label{fig:asn1_validation}
\end{center}

\section{Experimental Methodology}

\subsection{Test Environment}

All experiments are conducted on an Intel Core i9-12900K workstation (16 cores, 32GB RAM) running Ubuntu 22.04 LTS. Software versions include Open5GS 2.7.0, srsRAN Project 24.04.1, ZeroMQ 4.3.4, and Wireshark 4.0.11. Network configuration uses loopback for core-RAN communication and TUN interfaces for UE data plane with IP allocation from 10.45.0.0/16.

\vspace{0.5em}

\begin{center}
    \includegraphics[width=\columnwidth]{testing_terminal.jpg}
    \captionof{figure}{Testing terminal}
    \label{fig:asn1_validation}
\end{center}


\subsection{Test Scenarios}

Three primary scenarios are evaluated:

\textbf{Scenario 1: Single UE Paging} validates basic functionality where UE1 establishes connection, transitions to RRC\_IDLE after 10-second inactivity, receives paging upon downlink traffic arrival, and re-establishes connection via PRACH.

\textbf{Scenario 2: Multi-UE Selective Paging} verifies paging selectivity where all three UEs are in RRC\_IDLE, only target UE (e.g., UE2) responds to its specific 5G-S-TMSI, and non-target UEs remain idle, repeated for each UE.

\textbf{Scenario 3: Rapid Sequential Paging} tests system stability under rapid paging load where all UEs receive paging within 1-second intervals, validating correct handling of overlapping procedures and PRACH collision resolution.
\vspace{0.5em}

\subsection{Monitoring Infrastructure}

Real-time monitoring scripts continuously parse gNB logs for RRC state transitions, NGAP paging messages, and PRACH detections with microsecond-precision timestamps. Multiple PCAP captures are performed simultaneously on N2 interface (NGAP), data plane (ogstun), and with SCTP-specific filters. Custom Wireshark Lua dissectors decode DLT\_USER frames for MAC-NR and RLC-NR layers.

\vspace{0.5em}

\begin{center}
    \includegraphics[width=\columnwidth]{monitor.jpg}
    \captionof{figure}{Monitoring_terminal}
    \label{fig:asn1_validation}
\end{center}
\vspace{0.5em}

\section{Results and Analysis}

\subsection{Single UE Paging Performance}

Experimental results demonstrate clean RRC\_CONNECTED to RRC\_IDLE transitions with proper resource release after 10-second inactivity timer expiry. Upon downlink traffic arrival, the complete paging flow from AMF decision to RRC broadcast completes in 8.3ms (±0.9ms standard deviation). UE response time from paging reception to PRACH transmission is 5.1ms (±0.7ms), and full reconnection time from PRACH to RRC\_CONNECTED is 29.7ms (±2.3ms), resulting in end-to-end delay of 40.3ms (±3.2ms).

\subsection{Multi-UE Paging Selectivity}

Table \ref{tab:selectivity} shows results from selective paging tests across 15 trials (5 per UE). The system achieves 100\% paging selectivity with only target UEs responding while non-target UEs remain in RRC\_IDLE. Log analysis confirms that all UEs receive the broadcast paging message (as expected on PCCH), but only the UE with matching 5G-S-TMSI initiates PRACH response, demonstrating correct identity verification and power-efficient operation.

\vspace{0.5em}

\begin{center}
    \includegraphics[width=\columnwidth]{gnuradio.jpg}
    \captionof{figure}{gnuradio interface}
    \label{fig:asn1_validation}
\end{center}

\begin{table}[t]
\centering
\caption{Multi-UE Paging Selectivity Results}
\label{tab:selectivity}
\begin{tabular}{@{}lccc@{}}
\toprule
\textbf{Target UE} & \textbf{Responded} & \textbf{Others Idle} & \textbf{Success Rate} \\ \midrule
UE1 (5 tests) & 5/5 & Yes & 100\% \\
UE2 (5 tests) & 5/5 & Yes & 100\% \\
UE3 (5 tests) & 5/5 & Yes & 100\% \\ \midrule
\textbf{Overall} & \textbf{15/15} & \textbf{Yes} & \textbf{100\%} \\ \bottomrule
\end{tabular}
\end{table}

\subsection{ASN.1 Decoding Validation}

The custom ASN.1 decoder successfully validates all captured paging messages with complete match between decoder output, Wireshark dissections, and gNB logs. For a sample paging message targeting UE2, the decoder correctly extracts 5G-S-TMSI (AMF Set ID=1, AMF Pointer=0, 5G-TMSI=0x00000002), PLMN (MCC=999, MNC=70), TAC=1, and confirms APER encoding validity. This bit-level verification provides confidence in protocol correctness beyond what log analysis alone can achieve.

\vspace{0.5em}

\begin{center}
    \includegraphics[width=\columnwidth]{two.jpeg}
    \captionof{figure}{ASN.1 PER decoding and validation output for NGAP paging message}
    \label{fig:asn1_validation}
\end{center}

\subsection{Performance Comparison}

Table \ref{tab:comparison} compares paging latency measurements with reported values from commercial deployments and other research platforms. The software-based implementation achieves paging latency (8.3ms) better than commercial urban deployments (10-15ms) and laboratory SDR setups (12-20ms), validating the fidelity of the ZeroMQ-based approach for control-plane research.

\begin{table}[t]
\centering
\caption{Paging Latency Comparison}
\label{tab:comparison}
\begin{tabular}{@{}lc@{}}
\toprule
\textbf{Deployment Type} & \textbf{Paging Latency} \\ \midrule
This Work (Software) & 8.3 ms \\
Commercial 5G (Urban) & 10-15 ms \\
Lab SDR Setup & 12-20 ms \\
LTE (Reference) & 15-25 ms \\ \bottomrule
\end{tabular}
\end{table}

\subsection{Edge Case Handling}

Testing of edge conditions demonstrates robust system behavior. When paging is triggered during ongoing RRC establishment, the gNB correctly suppresses redundant paging and delivers data directly after connection completes. Custom paging messages with non-existent 5G-S-TMSI values result in no UE responses (all UEs correctly ignore mismatched identities), with proper timeout handling and no false positives observed.




\section{Evaluation and Discussion}

\subsection{Validation Methodology}

Unlike conventional demonstrations relying solely on log inspection, this work employs a multi-layered validation framework: (1) real-time monitoring with live event detection across gNB and UE logs, (2) PCAP protocol analysis using Wireshark with custom dissectors, (3) ASN.1 bit-level decoding for standards compliance, (4) cross-layer correlation matching events across NAS/NGAP/RRC/MAC/PHY, and (5) temporal analysis with microsecond-precision timestamping. This comprehensive approach provides significantly stronger experimental rigor than single-source validation methods.

\subsection{Advantages of Software Approach}

The ZeroMQ-based implementation offers several key advantages: (1) \textit{deterministic behavior} through elimination of RF-related variables like fading and interference, (2) \textit{complete protocol visibility} with full packet capture at all interfaces without proprietary encryption, (3) \textit{rapid iteration} enabling instant deployment and automated testing without hardware setup, and (4) \textit{reproducibility} with version-controlled configurations ensuring consistent experimental conditions. These benefits make the platform ideal for protocol research and education.

\subsection{Limitations}

The software approach abstracts certain physical layer phenomena including realistic RF propagation, actual MIMO spatial processing, and over-the-air interference scenarios. However, these limitations do not affect control-plane protocol correctness, which is the primary focus of this work. Current implementation supports 3 concurrent UEs but the modular architecture enables straightforward scaling with additional computational resources.

\subsection{3GPP Compliance}

Comprehensive compliance verification against 3GPP specifications confirms correct implementation of: paging message structure (TS 38.413 §8.2.2), 5G-S-TMSI format (TS 38.413 §9.3.3.4), TAI list encoding (TS 38.413 §9.3.3.10), RRC paging message (TS 38.331 §6.2.2), paging occasion calculation (TS 38.304), NGAP procedures (TS 38.413 §8.2), RRC state transitions (TS 38.331 §5.3), and PRACH procedure (TS 38.321 §5.1). No significant deviations from specifications were observed.

\section{Applications and Future Work}

\subsection{Educational Value}

This platform serves as an excellent educational tool for protocol engineering courses, providing hands-on experience with 3GPP specifications, ASN.1 encoding/decoding practice, and real-world protocol stack implementation. For 5G network architecture courses, it demonstrates Service-Based Architecture, control/user plane separation, and network function interactions through guided laboratory exercises with progressive complexity.

\subsection{Research Opportunities}

The platform enables various research directions including protocol optimization studies (paging efficiency for massive IoT, energy-optimal DRX cycle selection, load distribution algorithms), security research (paging privacy protection, DoS attack resistance, authentication evaluation), and network slicing (slice-specific paging strategies, QoS-aware prioritization).

\subsection{Future Enhancements}

Planned extensions include: (1) RRC\_INACTIVE state support with RNA-based paging and context resume optimization, (2) multi-cell deployment with inter-gNB paging coordination and handover scenarios, (3) 5G-Advanced features including network slicing with slice-specific paging and RedCap UE support, (4) machine learning integration for predictive paging and intelligent DRX adaptation, and (5) cloud-native deployment with containerization and distributed multi-node setup.

\section{Conclusion}

This work presents a comprehensive, protocol-accurate 5G NR multi-UE paging system implemented entirely in software, addressing critical gaps in 5G education and research. The integration of Open5GS core network, srsRAN Project RAN, and custom ASN.1 validation framework provides unprecedented observability into paging mechanisms. Experimental results demonstrate 100\% paging selectivity across multiple UEs, correct RRC state management, and performance comparable to commercial deployments (8.3ms paging latency). The platform's modular architecture, complete protocol visibility, and rigorous validation methodology make it valuable for advanced education, protocol research, and pre-deployment testing. By eliminating hardware dependencies while preserving authentic protocol behavior, this work democratizes 5G research and enables broader participation in next-generation wireless system development. The reproducible, research-grade testbed serves as a foundation for studying control-plane optimization, security mechanisms, and emerging 5G-Advanced features.

\section*{Acknowledgment}

The author thanks the FWC COMET Foundation and IIIT Bangalore for their support. Appreciation to the open-source communities behind Open5GS, srsRAN, and asn1c, and to 3GPP working groups for maintaining comprehensive specifications.

\begin{thebibliography}{9}

\bibitem{3gpp_arch}
3GPP TS 23.501, ``System architecture for the 5G System (5GS); Stage 2 (Release 17),'' 3rd Generation Partnership Project, 2023.

\bibitem{3gpp_procedures}
3GPP TS 23.502, ``Procedures for the 5G System; Stage 2 (Release 17),'' 3rd Generation Partnership Project, 2023.

\bibitem{3gpp_rrc}
3GPP TS 38.331, ``NR; Radio Resource Control (RRC); Protocol specification (Release 17),'' 3rd Generation Partnership Project, 2023.

\bibitem{3gpp_ngap}
3GPP TS 38.413, ``NG-RAN; NG Application Protocol (NGAP) (Release 17),'' 3rd Generation Partnership Project, 2023.

\bibitem{asn1_per}
ITU-T Recommendation X.691, ``Information technology - ASN.1 encoding rules: Specification of Packed Encoding Rules (PER),'' International Telecommunication Union, 2021.

\bibitem{asn1c}
L. Walkin, ``asn1c - Open Source ASN.1 Compiler,'' Available: \url{https://github.com/vlm/asn1c}, 2024.

\bibitem{open5gs}
Open5GS Project, ``Open5GS: Open Source implementation for 5G Core and EPC,'' Available: \url{https://open5gs.org}, 2024.

\bibitem{srsran}
srsRAN Project, ``srsRAN Project Documentation,'' Software Radio Systems, Available: \url{https://docs.srsran.com}, 2024.

\bibitem{paging_study}
M. Polese et al., ``Improved Handover Through Dual Connectivity in 5G mmWave Mobile Networks,'' \textit{IEEE J. Sel. Areas Commun.}, vol. 35, no. 9, pp. 2069-2084, Sep. 2017.

\end{thebibliography}

\end{document}